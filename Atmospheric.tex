\documentclass{article}
% Horizontal Magnetic Dipole over a lossy half-space
\usepackage[utf8]{inputenc} % Use it to include other characters than ABC
\usepackage[cmex10]{amsmath}
\usepackage{mdwmath}
\usepackage{mdwtab}
\usepackage{hyperref}
\usepackage{physics} % For using the oridnary derivative nomenclature
\usepackage{datetime} % Insert date and time
\usepackage{mathptmx} % Times new Roman

% ------------------------------- Useful Tricks Learnt
% Use ={}& to align subequations to the left
% Use = for single equations  %

% ----------------- To compile with references use the following order in Shell"
% 1. pdflatex filename.tex
% 2. bibtex filename (no extension)
% 3. bibtex filename (no extension)
% 4. pdflatex filename.tex
% -----------------
\begin{document}
  \title{\textsc{Atmospheric Absorption of Terahertz waves}\\}
  \date{\footnote{Last Modified: \currenttime, \today.}}
  \maketitle

  The idea of terahertz communication appears fascinating owing primarily to great bandwidth at hand. By the historic Shannon's capacity theorem, \cite{shannon1949communication} the data rate, C of a communication medium is related to the bandwidth, W as:
  \begin{equation}
    C = W \log_2 \left( 1 + \frac{S}{N} \right )
    \label{eq:shannon}
  \end{equation}
  where $\frac{S}{N}$ is the signal-to-noise ratio of the communicating wave. As it is evident in \ref{eq:shannon}, higher bandwidths lead to higher data rates. However, the applicability of terahertz waves for communication is hampered by the significant influence of the atmosphere. For frequencies greater than $ 1 THz$, signals face attenuation due to wave absorption due to water vapors and oxygen present in the atmosphere\cite{danylov2006thz}.

  \bibliography{mylib}
  \bibliographystyle{ieeetr}

  % \bibliographystyle{IEEEtran}
  % % argument is your BibTeX string definitions and bibliography database(s)
  % \bibliography{mylib}

\begin{equation}
  \delta_r = \sinr
\end{equation}

  \end{document}
