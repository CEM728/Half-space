\documentclass{article}
% Horizontal Magnetic Dipole over a lossy half-space
\usepackage[utf8]{inputenc} % Use it to include other characters than ABC
\usepackage[cmex10]{amsmath}
\usepackage{mdwmath}
\usepackage{mdwtab}
\usepackage{hyperref}
\usepackage{physics} % For using the oridnary derivative nomenclature
\usepackage{datetime} % Insert date and time
\usepackage{mathptmx} % Times new Roman

% ------------------------------- Useful Tricks Learnt
% Use ={}& to align subequations to the left
% Use = for single equations
%
% ----------------- To compile with references use the following order in Shell"
% 1. pdflatex filename.tex
% 2. bibtex filename (no extension)
% 3. bibtex filename (no extension)
% 4. pdflatex filename.tex
% -----------------

\begin{document}
  \title{\textsc{Integral Formulation of a Horizontally Oriented Magnetic Dipole in a layered medium}\\}
  \date{\footnote{Last Modified: \currenttime, \today.}}
  \maketitle

  We model the current distribution in a two-dimensional electron gas (2DEG) at a GaN/AlGaN heterostructure with a horizontal magnetic dipole (HMD) embedded in a layered medium as illustrated in \ref{fig:illustration}.
  \section{Green's Function Formulation}

  We consider a stratified medium with three layers with the dipole placed at the origin and the layered labeled as $0$ with dielectric constant $\varepsilon_0$, different from the free-space permittivity. We consider two semi-infinite layers, above (Region 1) at $z=d_1$ and below (Region -1) the source layer at $z=d_{-1}$ that extend to infinity. Assuming a TM case, the fields can be described by the electric field longitudinal component, $E_z$,
  \begin{equation}
    E_z = \int_{-\infty}^{\infty} E_z(\rho) dk_{\rho}
    \label{eq:Ez}
  \end{equation}

  In region $i$, the $E_z$ field can be written as:

  \begin{equation}
    E_z^i = \int_{-\infty}^{\infty} \left[ A_i e^{j k_{iz}z} + B_i e^{-j k_{iz}z} \right] H_n^{(1)}(k_{\rho}\rho) C_n(\phi) dk_{\rho}
    \label{eq:TM_Ez}
  \end{equation}

  where, $H_n^{(1)}(k_{\rho}\rho)$ is the nth-order Hankel function of the first kind and $C_n(\phi)$ is $\phi$ based function depending on the order of Hankel function and the dipole configuration. The remaining electric and magnetic field components can be found by the following Maxwell's equations:

  \begin{subequations}
    \begin{align}
      E_{\rho} =  \frac{1}{k_{\rho}^2} \frac{\partial}{\partial \rho} \frac{\partial E_z(k_\rho)}{\partial z}
      \label{eq:E_rho} \\
      E_{\phi} =  \frac{1}{k_{\rho}^2} \frac{1}{\rho} \frac{\partial}{ \partial \phi} \frac{\partial E_z(k_\rho)}{\partial z}
      \label{eq:E_phi}
    \end{align}
    \label{eq:E_fields}
  \end{subequations}

  \begin{subequations}
    \begin{align}
      H_{\rho} = -\frac{j\omega \varepsilon_i}{k_{\rho}^2} \frac{1}{\rho}\frac{\partial E_z(k_\rho)}{\partial \phi}
      \label{eq:H_rho} \\
      H_{\phi} =  +\frac{j\omega \varepsilon_i}{k_{\rho}^2} \frac{1}{\rho} \frac{\partial E_z(k_\rho)}{\partial \rho}
      \label{eq:H_phi}
    \end{align}
    \label{eq:H_fields}
  \end{subequations}

  The unknowns $A_i$ and $B_i$ can be found by applying boundary conditions at the two interfaces at $z = d$ and $z=-d$ that require the continuity of tangential components of the electric and magnetic fields. From (\ref{eq:TM_Ez}) and (\ref{eq:E_phi}), we obtain:

  \begin{equation}
    k_z^i \left( A_i e^{jk_z^i d_i} - B_i e^{-jk_z^i d_i}  \right) = k_z^{i-1} \left( A_{i-1} e^{jk_z^{i-1} d_i} - B_{i-1} e^{-jk_z^{i-1} d_i}  \right)
    \label{eq:E_rho_BC}
  \end{equation}

  \begin{equation}
    \varepsilon^i \left( A_i e^{jk_z^i d_i} - B_i e^{-jk_z^i d_i}  \right) = \varepsilon^{i-1} \left( A_{i-1} e^{jk_z^{i-1} d_i} - B_{i-1} e^{-jk_z^{i-1} d_i}  \right)
    \label{eq:H_rho_BC}
  \end{equation}

  In the region 0, the position of the observation point above or below the source leads to different values of unknowns $A_0$ and $B_0$. To make it distinct, for $z>0$,the unknowns are written as $A^>$ and $B^>$ and for $z<0$, $A^<$ and $B^<$ respectively. For HMD, the following conditions are used [\cite{kong1990electromagnetic}].

  \begin{subequations}
    \begin{align}
      A_0^> = A_{HMD} + E_{HMD} \\
      A_0^< = A_{HMD}
      \label{eq:A_0} \\
      B_0^> = B_{HMD} \\
      B_0^< = B_{HMD} + E_{HMD}
      \label{eq:B_0}
    \end{align}
    \label{eq:HMD}
  \end{subequations}

  where,
  \begin{equation}
    E_{HMD} = \frac{p_m \omega \mu_0 k_{\rho}^2}{8 \pi k_{0z}}
    \label{eq:E_HMD}
  \end{equation}

  \begin{equation}
    H_{HMD} = -\frac{p_m k_{\rho}^2}{8 \pi}
    \label{eq:H_HMD}
  \end{equation}

  The longitudinal propagation constant in region $i$ is given by:
  \begin{equation}
    k_z^i = \sqrt{k_i^2 - k_{\rho}^2}
    \label{eq:k}
  \end{equation}

  Now using (\ref{eq:E_rho_BC}) and (\ref{eq:H_rho_BC}), at $z=d_1$, we have:
  \begin{equation}
    k_z^1 \left( A_1 e^{jk_z^1 d_1} - B_1 e^{-jk_z^1 d_1}  \right) = k_z^{0} \left( A_{0}^> e^{jk_z^{0} d_1} - B_{0}^> e^{-jk_z^{0} d_1}  \right)
    \label{eq:E_rho_d1}
  \end{equation}

  \begin{equation}
    \varepsilon^1 \left( A_1 e^{jk_z^1 d_1} - B_1 e^{-jk_z^1 d_1}  \right) = \varepsilon^{0} \left( A_{0}^> e^{jk_z^{0} d_1} - B_{0}^> e^{-jk_z^{0} d_1}  \right)
    \label{eq:H_rho_d1}
  \end{equation}

  After adding (\ref{eq:E_rho_d1}) and (\ref{eq:H_rho_d1}), we get,




  \bibliography{mylib}
  \bibliographystyle{ieeetr}
\end{document}
