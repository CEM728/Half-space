\documentclass{article}
% Horizontal Magnetic Dipole over a lossy half-space
\usepackage[cmex10]{amsmath}
\usepackage{mdwmath}
\usepackage{mdwtab}
\begin{document}
  We consider a two-dimensional problem of a magnetic dipole placed along the interface of free-space and lossy dielectric as illustrated in \ref{fig:setup}. The Green's function of the problem can be formulated by first expressing the magnetic line current polarized along the y-axis and located at the interface, $z = 0$:

  \begin{equation}
    \overrightarrow{M} = m \delta(x) \delta(z) \mathbf{y}
    \label{Current}
  \end{equation}

  For the given problem, region in the lower half-space, $(z < 0)$ is composed of a lossy dielectric, with complex relative permittivity, $\varepsilon_1 = \varepsilon' - i\varepsilon''$ whereas the upper half-space, $(z > 0)$ is free space. The magnetic vector potential equations in the two regions are given as:

  \begin{subequations}
    \begin{equation}
      (\nabla^2 + k_1^2) F_1 = \delta,
      \label{eq:potential_lower}
    \end{equation}
    \begin{equation}
      \nabla\cdot{\bf H} = 0,
      \label{eq:potential_upper}
    \end{equation}
  \end{subequations}

As I said earlier that this is utter nonsense

\end{document}
