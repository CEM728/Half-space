\documentclass{IEEEtran}
% Horizontal Magnetic Dipole over a lossy half-space
\usepackage[cmex10]{amsmath}
\usepackage{mdwmath}
\usepackage{mdwtab}
% \usepackage{IEEEtrantools}

\begin{document}

We consider the problem of a horizontally oriented magnetic line source located above a lossy dielectric half-space in air. We assume the source to be time-harmonic and y-directed as shown in \ref{fig: illustration}. The lossy dielectric characterized by a complex dielctric constant ($\varepsilon_b$) exists in the region $z < 0$. For simplicity, we assume that the source lies at a height $z = h$ above the interface, $z = 0$ and is expressed as:

\begin{IEEEeqnarray}{rCl}
  \overrightarrow{M} = \mathcal{I}_m \delta(x) \delta(z -h) \mathbf{y}
  \label{Current}
\end{IEEEeqnarray}

where $\mathcal{I}_m$ is the amplitude of the source.

For the two-dimensional problem at hand, we write the scalar Helmholtz equations for the respective media.

\begin{IEEEeqnarray}{LCL}
    \left( \nabla_t^2 + k_a^2 \right) F_y^a = -\varepsilon_0 \mathcal{I}_m  \delta(x) \delta(z - h), z > 0 \IEEEyessubnumber*\\
    \left( \nabla_t^2 + k_b^2 \right) F_y^b = 0,     z <= 0.
\end{IEEEeqnarray}

where $\nabla_t^2$ is the Laplacian in the transverse direction to the source (zx-plane). In terms of the magnetic vector potential, the electric and magnetic fields are given by:
\begin{IEEEeqnarray}{rCl}
  \overrightarrow{E}  = \frac{-1}{\varepsilon} \nabla \times \overrightarrow{F} \IEEEyessubnumber*\\
  \overrightarrow{H}  = \frac{-j\omega}{k^2} \left( k^2 + \nabla \nabla \cdot \right) \overrightarrow{F}
\end{IEEEeqnarray}

The boundary conditions extracted from the continuity of the tangential fields at the interface $z = 0$ are:

\begin{IEEEeqnarray}{rCl}
    \frac{F_y^a}{\varepsilon_0} = \frac{F_y^a}{\varepsilon_0} \IEEEyessubnumber*\\
    \frac{\partial F_y^a}{\partial y} = \frac{\partial F_y^b}{\partial y}
\end{IEEEeqnarray}


  \end{document}
