\documentclass{article}
% Horizontal Magnetic Dipole over a lossy half-space
\usepackage[cmex10]{amsmath}
\usepackage{mdwmath}
\usepackage{mdwtab}

\begin{document}

We consider the problem of a horizontally oriented magnetic line source located above a lossy dielectric half-space in air. We assume the source to be time-harmonic and y-directed as shown in \ref{fig: illustration}. The lossy dielectric characterized by a complex dielctric constant ($\varepsilon_a$) exists in the region $z < 0$. For simplicity, we assume that the source lies at a height $z = h$ above the interface, $z = 0$ and is expressed as:

\begin{equation}
  \overrightarrow{M} = \mathcal{I}_m \delta(x) \delta(z -h) \mathbf{y}
  \label{Current}
\end{equation}

where $\mathcal{I}_m$ is the amplitude of the source.

For the two-dimensional problem at hand, we write the scalar Helmholtz equations for the respective media.
\begin{subequations}
  \begin{equation}

    \left( \nabla_t^2 + k_0^2 \right) F_y = -\varepsilon_0 \mathcal{I}_m \delta(x) \delta(z -h), z > 0,
    \label{eq:potential_lower}

  \end{equation}

  \begin{equation}

    \left( \nabla_t^2 + k_1^2 \right) F_y = 0, z <= 0,
    \label{eq:potential_lower}

  \end{equation}
\end{subequations}

  \end{document}
