\documentclass{IEEEtran}
% Horizontal Magnetic Dipole over a lossy half-space
\usepackage[cmex10]{amsmath}
\usepackage{mdwmath}
\usepackage{mdwtab}
\usepackage{physics} % For using the oridnary derivative nomenclature
% \usepackage{IEEEtrantools}

\begin{document}

  We consider the problem of a horizontally oriented magnetic line source located above a lossy dielectric half-space in air. We assume the source to be time-harmonic and z-directed as shown in \ref{fig:illustration}. The lossy dielectric characterized by a complex dielctric constant ($\varepsilon_b$) exists in the region $y < 0$. For simplicity, we assume that the source lies at a height $y = h$ above the interface, $y = 0$ and is expressed as:

  \begin{equation}
    \overrightarrow{M} = \mathcal{I}_m \delta(x) \delta(y -h) \widehat{z}
    \label{eq:Current}
  \end{equation}

  where $\mathcal{I}_m$ is the amplitude of the source.

  For the two-dimensional problem at hand, we write the scalar Helmholtz equations for the respective media.

  \begin{subequations}
    \begin{equation}
      \left( \nabla_t^2 + k_a^2 \right) F_z^a = -\varepsilon_0 \mathcal{I}_m  \delta(x) \delta(y - h), y > 0
      \label{eq:Hemup}
    \end{equation}
    \begin{equation}
      \left( \nabla_t^2 + k_b^2 \right) F_z^b = 0,     y <= 0
      \label{eq:Hemdn}.
    \end{equation}
    \label{Hem}
  \end{subequations}

  where $\nabla_t^2$ is the Laplacian in the transverse direction to the source (xy-plane). In terms of the magnetic vector potential, the electric and magnetic fields are given by:

  \begin{subequations}
    \begin{equation}
      \overrightarrow{E}  = \frac{-1}{\varepsilon} \nabla \times \overrightarrow{F} \label{eq:E}\\
    \end{equation}
    \begin{equation}
      \overrightarrow{H}  = \frac{-j\omega}{k^2} \left( k^2 + \nabla \nabla \cdot \right) \overrightarrow{F} \label{eq:H}
    \end{equation}
  \end{subequations}

  The boundary conditions extracted from the continuity of the tangential fields at the interface $y = 0$ are:

  \begin{subequations}
    \begin{equation}
      \widetilde{F}_z^a = \widetilde{F}_z^b
      \label{eq:HBC}\\
    \end{equation}
    \begin{equation}
      1 /\varepsilon_0 \frac{\partial \widetilde{F}_z^a}{\partial y} = 1 /\varepsilon_b\frac{\partial \widetilde{F}_z^b}{\partial y}
      \label{eq:EBC}
    \end{equation}
  \end{subequations}

  The $\sim$ in the preceding equations indicates that the magnetic potential has been Fourier transformed in one dimension from $x$ to $k_x$. Eqs. \ref{eq:Hemup} and \ref{eq:Hemdn} are transformed to:

  \begin{subequations}
    \begin{equation}
      \left( \dv[2]{}{y} + (k_a^2 - k_y^2) \right) \widetilde{F}_z^a = -\varepsilon_0 \mathcal{I}_m  \delta(x) \delta(y - h), y > 0
      \label{eq:FTHemup}\\
    \end{equation}
    \begin{equation}
      \left( \dv[2]{}{y} + (k_b^2 - k_y^2) \right) \widetilde{F}_z^b = 0,     y <= 0 \label{eq:Hemdn}
    \end{equation}
  \end{subequations}




\end{document}
